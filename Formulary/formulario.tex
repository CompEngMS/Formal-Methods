\documentclass{article}
\usepackage[utf8]{inputenc}

\title{Formal Methods Formulas and Algorithm}
\author{Nicola Di Santo }
\date{July 2020}

\usepackage[a4paper,margin=1.5cm]{geometry}
\usepackage[ruled,linesnumbered]{algorithm2e}
\usepackage{natbib}
\usepackage{graphicx}
\usepackage{amsmath}
\usepackage{setspace}
\DeclareRobustCommand{\impl}{\text{\reflectbox{$\subset$}}}


\begin{document}

\maketitle
\section{Introduciton}
this document is a collection of formulas to use during formal methods exercises. The .tex file is available to make it always richer, bigger and better. 

\twocolumn

\section{FOL}
\textbf{$\alpha$-rules}:\newline
$\frac{\phi \wedge \psi}{\substack{\phi\\\psi}} $ 
$\frac{\neg(\phi \vee \psi)}{\substack{\neg\phi\\\neg\psi}} $ 
$\frac{\neg(\phi \impl \psi)}{\substack{\phi\\\neg\psi}} $
$\frac{\neg\neg\phi}{\phi} $ \\

\textbf{$\beta$-rules}:\newline
$\frac{\phi \vee \psi}{\phi|\psi} $ 
$\frac{\neg(\phi \wedge \psi)}{\neg\phi|\neg\psi} $ 
$\frac{\phi \impl \psi}{\neg\phi|\psi} $ \\

\textbf{extra-rules}:\newline
$\frac{\substack{\phi\\\neg\phi}}{X} $ 
$\frac{\phi \equiv \psi)}{\substack{\phi|\neg\phi\\\psi|\neg\psi}} $ 
$\frac{\neg(\phi \equiv \psi)}{\substack{\neg\phi\\\psi}\substack{|\\|} \substack{\phi\\\neg\psi}} $ \\

\textbf{$\delta$-rules}:\newline
$\frac{\forall x.\phi(x) }{\phi(t)} $ 
$\frac{\neg\exists x.\phi(x) }{\neg\phi(t)} $\\

\textbf{$\gamma$-rules}:\newline
$\frac{\neg\forall x.\phi(x) }{\neg\phi(c)} $ 
$\frac{\exists x.\phi(x) }{\phi(c)} $\\

\textbf{Tableaux}: \\
prove $\phi\equiv\psi$: check $\neg(\phi\equiv\psi)$ is UNSAT\newline
$\phi$ is valid: $\neg\phi$ is UNSAT\newline
$\Gamma\models\phi$: check $\Gamma\cup\neg\phi$ closes (UNSAT)\newline
$\Gamma$ SAT: check for an open branch of $\Gamma$\newline
In general each existential is a fresh new constant while universal can be any term also the constant of the existential. To clash instantiate existential and then use universal to make the clash.

\section{UML TO FOL}
a is the attribute name,A the association name, C is the class name, T is the type name, i and j the multiplicities, P is type name for parameter, R is type name for return value.
Class: 
\newline 
$\forall x,y. a(x,y) \impl C(x) \wedge T(y)$ \newline 
$\forall x. C(x) \impl i\leq\{y|a(x,y)\}\leq j $\\

Association: 
\newline $\forall x_1,...,x_n . A(x_1,...,x_n) \impl C(x_1) \wedge ... C(x_n)$ \newline 
$\forall x_1. C(x_1) \impl i\leq\{x2|a(x_1,x_2,...)\}\leq j $\newline
$\forall x_2. C(x_2) \impl i\leq\{x1|a(x_1,x_2,...)\}\leq j $ \\

Generalization: 
\newline $\forall x. C_i(x) \impl C(x) for i=1,...,n $ (is-a) \newline 
$\forall x. C_i(x) \impl \neg C_j(x) for i\not=j $ (disjoint)\newline
$\forall x. C(x) \impl C_1 \vee.. \vee C_i(x)...\vee C_n(x)  $ (completeness) \\

Subset:\newline
$\forall x,y . assoc_1(x,y)\impl assoc_2(x,y)$ can refine mult. \\

Association Class: \newline
$\forall x,y,z. a(x,y,z) \impl A(x,y)\wedge T(z)$\newline
$\forall x,y  A(x,y) \impl \forall z. i \leq \{z|a(x,y,z)\}\leq j$ + assoc mult.\\

Reification (when assoc is a class+key constr):\newline
$\forall x,y r_i(x,y) \impl A(x) \wedge C_i(y) for i=1,..,n$\newline
$\forall x  A(x) \impl \exists y. r_i(x,y) for i=1,..,n$\newline
$\forall x,y,y'  r_i(x,y) \wedge r_i(x,y') \impl y=y' for i=1,..,n$\newline
$\forall y_1,...,y_n,x,x'. \wedge_{i=1}^{n} r_i(x,y_i)\wedge r_i(x',y_i)  \impl x=x'$\\

Methods: \newline
\sloppy %to break formula into multiline
$\forall x,p_1,...,p_m,r . f_{C,P_1,...,P_m}(x,p_1,...,p_m,r) \impl C(x) \wedge (\wedge_{i=i}^{m}P_i(p_i))\wedge R(r) $ \newline
$\forall x,p_1,...,p_m,r,r' . f_{C,P_1,...,P_m}(x,p_1,...,p_m,r) \wedge f_{C,P_1,...,P_m}(x,p_1,...,p_m,r') \impl r=r' $ \\

\section{Evaluation Semantic}

$\frac{(a,s)\rightarrow s'}{true}$ if $s \models Pre(a) \wedge s'\models Post(a,s)$\\ [2mm]
$\frac{(skip,s)\rightarrow s}{true}$ \\[2mm]
$\frac{(\delta_1;\delta_2,s_0)\rightarrow s_f}{(\delta_1,s_0)\rightarrow s_1 \wedge (\delta_2,s_1)\rightarrow s_f }$ \\[2mm]
$\frac{( if (\phi) then\{ \delta_1\} else \{ \delta_2\} ,s)\rightarrow s'}{(\delta_1,s)->s'}, s\models\phi$ \\[2mm]
$\frac{( if (\phi) then \{ \delta_1\} else \{ \delta_2\} ,s)\rightarrow s'}{(\delta_2,s)->s'}, s\not\models\phi$ \\[2mm]
$\frac{(while (\phi) do\{\delta\}, s)\rightarrow s'}{(\delta,s)\rightarrow s'' \wedge (while \phi do\{\delta\}, s'')\rightarrow s' }, s\models\phi$\\[2mm]
$\frac{(while (\phi) do\{\delta\}, s)\rightarrow s'}{true}, s\not\models\phi$\\

\section{Transition Semantic}
\subsection{Transition Rules}
$\frac{(a,s)\rightarrow(\epsilon,s')}{true}$ if $s \models Pre(a) \wedge s'\models Post(a,s)$ \\[2mm]
$\frac{(skip,s)\rightarrow(\epsilon,s')}{true}$ \\[2mm]
$\frac{(\delta_1;\delta_2,s)\rightarrow(\delta_1';\delta_2,s')}{(\delta_1,s)\rightarrow(\delta_1',s')}$ \\[2mm]
$\frac{(\delta_1;\delta_2,s)\rightarrow(\delta_2',s')}{(\delta_2,s)\rightarrow(\delta_2',s')}$ if $(\delta_1,s)^{\surd}$\\[2mm]
$\frac{(if(\\phi)then\{\delta_1\}else\{\delta_2\},s)\rightarrow(\delta_1',s')}{ (\delta_1,s)\rightarrow(\delta_1',s')}$ if $s\models\phi$ \\[2mm]
$\frac{(if(\\phi)then\{\delta_1\}else\{\delta_2\},s)\rightarrow(\delta_2',s')}{ (\delta_2,s)\rightarrow(\delta_2',s')}$ if $s\not\models\phi$ \\[2mm]
$\frac{(while(\phi)do\{\delta\},s)\rightarrow(\delta';while(\phi)do\{\delta\},s')}{\delta,s)\rightarrow(\delta',s')} if s\models\phi$
\subsection{Termination Rules}
$\frac{(\epsilon,s)^{\surd}}{true}$\\[2mm]
$\frac{(\delta_1;\delta_2,s)^{\surd}}{(\delta_1,s)^{\surd}\wedge(\delta_2,s)^{\surd}}$\\[2mm]
$\frac{(if(\\phi)then\{\delta_1\}else\{\delta_2\},s)^{\surd}}{(\delta_1,s)^{\surd}}$\\[2mm]
$\frac{(if(\\phi)then\{\delta_1\}else\{\delta_2\},s)^{\surd}}{(\delta_2,s)^{\surd}}$\\[2mm]
$\frac{(while(\phi)do\{\delta\},s)^{\surd}}{true}, if s\models\neg\phi$\\[2mm]
$\frac{(while(\phi)do\{\delta\},s)^{\surd}}{(\delta,s)^{\surd}}, if s\models\phi$\\[2mm]
\section{Hoare Logic}
$P\Rightarrow I$ \newline
$(\neg g \wedge I)\Rightarrow Q$ \newline
$\{g \wedge I\}S\{I\}$: find wp of S and check if $(g \wedge I)\Rightarrow wp$

\section{CTL TO $\mu$-CALC}
$EX p: <->p$\\
$AX p: [-]p$\\
$EF p: \mu X.p\vee<->X $\\
$AFp: \mu X.p\vee[-]X $\\
$pEUq: \mu X. q\vee p \wedge <->X $\\
$pAUq: \mu X. q\vee p \wedge [-]X $\\
$EGp: \nu X.p\wedge<->X$\\
$AGp: \nu X.p\wedge[-]X$\\

\section{Conjunctive Queries}

	\begin{algorithm}
		\KwIn{q: conj query}
		$\Delta^{I_q}$ = all constant and variable of q \;
		$P^{I_q}={(t_1,...,t_n),...} $for all $P_i(t_1,...,t_n)$ in q\;
		$c^{I_q}=c$ c is in q \;
		\caption{Canonical Interpretation}
	\end{algorithm}

	\begin{algorithm}
		\KwIn{q: conj query\\I: interpretation (DB)}
		write q in canonical Interp.\;
		find assignment $\alpha(.)$ for each variable in q\;
		assign to each constant itself: $\alpha(c)=c$ \;
		\Return $\alpha$ as the homomorphism between I and $I_q$ if exist or $I\not\models q$\;
		\caption{$I\models q$ ($\exists h(I_q)=I$)}
	\end{algorithm}

	\begin{algorithm}
		\KwIn{$q_1$: conj query\\$q_2$: conj query\\
	show $q_1 \subseteq q_2$i.e. $q_2 \Rightarrow q_1$}
		\Begin(containement){
			freeze all variable by assigning a constant: assume $q_1(x,y)\leftarrow e(x,y,z)...$ you must freeze only the one in the argument $q_1(c_1,c_2)\leftarrow e(c_1,c_2,z)...$ \;
			build $I_{q_1}$\;
			check if db tables of $I_{q_1}$ are true in $q_2$: find an assignment to $q_2$ variables that makes $I_{q_1}\models q_2$ by guessing \;
		}
		\Begin(homomorphism){
			\tcc{To verify the homomorphism check $I_{q_1}\models q2$ iff $\exists h.I_{q_2}\Rightarrow I_{q_1} $} 
			compute $I_{q_2}$ as interpretation of $q_2$ \;
			compute mapping from object o$I_{q_2}$ to $\delta^{I_{q_1}}$ \;
			check if mapping holds also for tuples \;
		}
		
		\caption{$q_1\subseteq q_2$}
	\end{algorithm}

	\begin{algorithm}
		\KwIn{q: conj query\\Incomplete DB}
		transform DB into $q_D$ where each null become an existentially quantified variable \;
		DB$\models q$ iff $q_D \subseteq q$ (see containment algorithm)\;
		
		\caption{Check if incomplete db $\models q$}
	\end{algorithm}


\section{Bisimilarity}

A state $s_0$ of transition system S is bisimilar, or simply equivalent, to a state $t_0$ of transition system T iff there exists a bisimulation between the initial states $s_0$ and $t_0$ (note: bisimilarity  the largest bisimulation). \\ A binary relation R is a bisimulation if $(s,t)\in R$ implies that s is final iff t is final and for all action a if $s\rightarrow_a s'$ then $\exists t'. t\rightarrow_a t' and (s',t')\in R $ if $t\rightarrow_a t'$ then $\exists s'. s\rightarrow_a s' and (s',t')\in R $ \\

\begin{algorithm}
	\KwIn{T, S}
	Compute R = $T\times S$ \;
	remove from R all tuples (s,t) where s is final and t is not and vice versa \;
	remove from R all tuples (s,t) where s can do $a_i$ and t cannot (and vice versa)\;
	remove from R all tuples ($s_i,t_j$) that can reach (s',t') but then (s',t') is not in R \;
	\Return R\;
	\caption{Check Bisimulation between T and S}
\end{algorithm}
\end{document}
